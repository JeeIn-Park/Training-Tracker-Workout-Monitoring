\chapter{Introduction}

about excel 9:10
about visualisation 9:30
references 10:00
survey data 10:30


\label{chap:context}
\noindent
In recent years there has been a surge in public awareness of the important role physical activity plays in maintaining health, which is reflected in the frequency of gym visits. From 2010 to 2019, there had been a 24\% increase in the number of individuals attending the gym over 100 times annually.

As public interest in fitness grows, the importance of monitoring workout progress becomes more apparent. It has been shown that the availability of quantifiable data can stimulate users' interest in enhancing their activity levels. This implies that providing tools to track this data supports the maintenance of motivation and commitment to fitness goals, which could lead to improved health outcomes over time.

The growth in recognition of training and tracking progression has led to significant expansion in the global fitness app market, as these apps allow users to monitor their workouts anytime and anywhere. In 2023, the market was valued at \$1.54 billion and is expected to grow at a compound annual growth rate (CAGR) of 17.7\% from 2024 to 2030. This trend is underscored by a marked rise in the number of fitness app users and downloads, with hundreds of millions of users and over 850 million downloads reported in 2023 alone. 

Numerous workout tracking apps have come of this market expansion, whose primary focuses tend to quantitative data. However, features limited to logging and analysing progress are inadequate for sustaining user adherence to physical activity goals. Identifying and integrating motivational factors into app design is crucial for enhancing user engagement and promoting consistent exercise routines.



people actually logs data a lot when they go to the gym (what percent of people logs data?) 
this is from the fact that quantifiable data can enhance motivation
fitness tracking apps targets these people. 
what is fitness tracking app : app that enable users to keep tracking on their workingout progress

found most of people log exercises are using apps or excels
there was high growth on fitness tracking app market.
altough high growth of fitness tracking app market, there are still many people use excel for logging. (what percentage of people are using excel among people who tracks data)
got an idea from this : 
what if I make the app which has both of their pros and reducing cons?

additionally, these logs are only focusing on quantifiable data.
but while I am actually tracking on my own workout progress,
I found  it is difficult to track on muscle status
I knew it is important to not work on same muscle before it is recovered to avoid injuries,
but it was almost impossible to track on when I used what muscles, and when each muscles would be recovered.
for this I needed to track on when I did what exercises, 
knowing what muscles are involved in each exercise,
and how long it passed after I worked on that exercise.



Assuming this happens because.... 
the reason why people are suing excels are...


the reason why people are using apps are...

the main aim of this project is increasing accesiblity of the app
with two main components 


2.



% TODO  : why muscle visualisation 


\begin{quote}
\noindent
The primary goal of this project is to create an Android application designed to effectively monitor workout routines and boost motivation.
More particularly, the solid aims are: 

\begin{enumerate}
\item Investigate factors that can extend user engagement and commitment to their physical goals and understand the role of the training tracking app in sustaining this motivation.
\item Design an application that addresses the possible shortcomings of workout apps, which can deter consistency, and enhance the benefits that such apps can offer to users.
\item Implement the app by utilising technical resources to improve the user experience.
\item Evaluate the app in terms of efficiency and user satisfaction and demonstrate that the proposed enhancements are beneficial.

\end{enumerate}
\end{quote}
