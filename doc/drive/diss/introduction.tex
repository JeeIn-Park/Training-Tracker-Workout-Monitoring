survey data 10:30
----------------------------------------


Many people check how their workouts are progressing through fitness tracking applications. These applications give an opportunity to keep track of workout routines, set goals, and later check on the progress. According to Curry (2024), the fitness app market has been on the rise since the valuation of \$1.54 billion in 2023, and the expected Compound Annual Growth Rate (CAGR) for the period of 2024 to 2030 is 17.7\% \cite{curry2024}. This has epitomized a continued reliance on digital tools in the management of fitness.

Despite this increased market for fitness apps, still the major portion of users uses Microsoft Excel to manage their health and fitness. The existence of easily available templates and resources in regards to health tracking in Excel explains its use and adaptability in this context. For example, there is an abundance of Excel templates on tracking diets, exercises, and numerous other activities regarding health \cite{microsoft2024}. This dual approach can be explained by the fact that providing quantifiable data is more motivational and encouraging to keep on the path of attaining the goals for one's fitness. The provision of such tracking tools will help increase the level of activity for the concerned users, and thus better health results may be witnessed with each passing day \cite{plos2024}.
However, these logs are majorly focused on quantitative data. During my personal experience with tracking workout progress, I found it very hard to understand and monitor the status of muscles effectively. It was almost impossible to track which muscles are used, and when each muscle group would be recovered. For this, it is essential to track the exact exercises done, the muscles involved, and the recovery time needed. This kind of complex orientation in the tracking of the status of muscles is very crucial in the avoidance of injuries. This is by making sure that the same muscle is not overworked when it is not fully recovered. People also tend to ignore certain muscles if they follow a routine since they lack a comprehensive tracking system. It is imperative to work on all muscles groups to reduce muscle imbalance and muscle loss.

From this introduction, it got me thinking what if I create an app that encompasses both the pros and cons of Excel and the training tracking app. What if I made an app that could automatically track the muscle state, based on the given logs of exercises, and show it visually to track the status of the muscle. The project actually is to make an Android application that will help track the workouts easily, along with quantifiable data to be shown in a visual way for easily tracking the muscle status.
More particularly, the solid aims are:

\begin{quote}
    \begin{enumerate}
\item Investigate factors that may extend user engagement and commitment to their physical goals, and use this data to design the first iteration of the app.
\item Develop an alternative version of the application that overcomes possible limitations of the workout tracking system based on spreadsheets. 
\item Implement the final version of the app based on actual user feedback. 
\item Assess the efficiency and satisfaction of the app and show that enhancements proposed are worthwhile. 
\end{enumerate}
\end{quote}



---------------------------------------
many people going to the gym track on their workout progress using the fitness tracking apps which are the apps that enable users to keep tracking on their workingout progress
"Fitness apps generated \$3.58 billion revenue in 2023, a 9.1\% increase on the year prior, and were downloaded over 850 million times, with 368 million users globally.
(Citation: Curry D. Fitness App Revenue and Usage Statistics (2024). Business of Apps. Updated March 26, 2024. Available from: https://www.businessofapps.com/data/fitness-app-market/)

this leaded huge growth on fitness tracking app market.  In 2023, the market was valued at \$1.54 billion and is expected to grow at a compound annual growth rate (CAGR) of 17.7\% from 2024 to 2030,
although significant expansion in the global fitness app market(Citation: Grand View Research. "Fitness App Market Size To Reach \$4.80 Billion By 2030." December 2023. Accessed May 14, 2024. Available from: https://www.grandviewresearch.com/industry-analysis/fitness-app-market),it's clear that many individuals utilize tools like Excel as well for health and fitness purposes
The evidence that many individuals utilize tools like Excel for health and fitness tracking is supported by the availability of numerous templates and resources dedicated to this purpose. For instance, Microsoft offers various Excel templates specifically designed for tracking diet, exercise, and other health-related activities. (citation : Microsoft. "Track your health and fitness goals in Excel." Microsoft Support. Accessed May 14, 2024. Available from: https://support.microsoft.com/en-us/office/track-your-health-and-fitness-goals-in-excel-93fbce1d-2969-4c11-81c6-00921b58d062)

It is because the availability of quantifiable data in both ways. It can stimulate users' interest in enhancing their activity levels. This implies that providing tools to track this data supports the maintenance of motivation and commitment to fitness goals, which could lead to improved health outcomes over time. (citation : https://journals.plos.org/digitalhealth/article?id=10.1371/journal.pdig.0000087)

however, these logs are only focusing on quantifiable data.
but while I am actually tracking on my own workout progress,
I found  it is difficult to track on muscle status
but it was almost impossible to track on when I used what muscles, and when each muscles would be recovered.
for this I needed to track on when I did what exercises, 
knowing what muscles are involved in each exercise,
and how long it passed after I worked on that exercise.
beacuse of this complex orientation on muscle status tracking, although it is essential to know it as  I shouldn't work on same muscle before it is recovered to avoid injuries, people don't do it precisely. 
Moreover, people are likely to not notice what muscles they haven't worked out if they just follow the usual routine.
But it is important to work on all muscles to minimise muscle loss. 

got an idea from this intro : 
what if I make the app which has both of their pros and reducing cons from excel and training tracking app?
what if I make an app which track on muscle state automatically from given exercise logs and visually present it for easy tracking on muscle status?
The primary goal of this project is to create an Android application designed to effectively monitor workout with quantifiable data with visualised muscle.
More particularly, the solid aims are: 

\item Investigate factors that can extend user engagement and commitment to their physical goals and use this data to make initial version of the app.
\item the second version of application that addresses the possible shortcomings of spread sheet workout tracking.
\item Implement the final version of app from actual users' feedback.
\item Evaluate the app in terms of efficiency and user satisfaction and demonstrate that the proposed enhancements are beneficial.


fitness tracking apps targets these people. 


found most of people log exercises are using apps or excels
there was high growth on fitness tracking app market.
altough high growth of fitness tracking app market, there are still many people use excel for logging. (what percentage of people are using excel among people who tracks data)

additionally,

---------------------------------------

In recent years there has been a surge in public awareness of the important role physical activity plays in maintaining health, which is reflected in the frequency of gym visits. From 2010 to 2019, there had been a 24\% increase in the number of individuals attending the gym over 100 times annually.

As public interest in fitness grows, the importance of monitoring workout progress becomes more apparent. 

The growth in recognition of training and tracking progression has led to  This trend is underscored by a marked rise in the number of fitness app users and downloads, with hundreds of millions of users and over 850 million downloads reported in 2023 alone. 

Numerous workout tracking apps have come of this market expansion, whose primary focuses tend to quantitative data. 

However, features limited to logging and analysing progress are inadequate for sustaining user adherence to physical activity goals. Identifying and integrating motivational factors into app design is crucial for enhancing user engagement and promoting consistent exercise routines.


people actually logs data a lot when they go to the gym (what percent of people logs data?) 
this is from the fact that quantifiable data can enhance motivation
fitness tracking apps targets these people. 
what is fitness tracking app : app that enable users to keep tracking on their workingout progress

found most of people log exercises are using apps or excels
there was high growth on fitness tracking app market.
altough high growth of fitness tracking app market, there are still many people use excel for logging. (what percentage of people are using excel among people who tracks data)

additionally, these logs are only focusing on quantifiable data.
but while I am actually tracking on my own workout progress,
I found  it is difficult to track on muscle status
but it was almost impossible to track on when I used what muscles, and when each muscles would be recovered.
for this I needed to track on when I did what exercises, 
knowing what muscles are involved in each exercise,
and how long it passed after I worked on that exercise.
beacuse of this complex orientation on muscle status tracking, although it is essential to know it as  I shouldn't work on same muscle before it is recovered to avoid injuries, people don't do it precisely. 
Moreover, people are likely to not notice what muscles they haven't worked out if they just follow the usual routine.
But it is important to work on all muscles to minimise muscle loss. 




\begin{quote}
\noindent


\end{enumerate}
\end{quote}
