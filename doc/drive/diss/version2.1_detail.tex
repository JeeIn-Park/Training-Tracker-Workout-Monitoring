
significant changes 

found most of people log exercises are using apps or excels
the reason why people are using apps are...
the reason why people are suing excels are...
employ benefits from excel and fitness tracking apps.
what is the benefit of excel?
what is disadvantages?
these disadvantages can be solved with fitness tracking app. 


focus on researched factors (user centered app)
enhancing user experiences
(minimalise steps that user needs to take to log their progress, user can customise set recording, long click to delete a card with warning)




section - app  Scalability  update 

1. string formatter 
when views are dealing with data and do formatting it to the form of string,
rather than deal with it in all separated classes,
I added the central object for string formatting, called FormattedStringGetter.
This class gives various ways for formatting data to string. 
receieving the data with selected method, this object returns the formatted string in the form you needed.
so passing the data and the form you want to format it, you can get the formatted string.
as all the string formatation is handled in this one object,
it became easier to deal with app consistency in terms of string formatation.
for example, if I want to make change on one specific data's formatation,
I can easily modify all the usage of this data when it is formatted to string.
just by modifying methods in this object which utilise that data. 

2. improved data management
additional components are introduced for better data management, followiing the factory design pattern.
These factories are responsible on generating object with given data class and data.
Alongside with default value setted in data class,
using the factories, I could deal with data generation from given information, specifying and dealing with cases.
I implemented what data needs to be generated in each case, 
depends on where it is used.
As all the data generation accross the app is dealed with factories,
I could easily deal with making changes on data class itself, or making change on usage.

Also, made changes on data storages.
I made saving methods private.
explain how and why I did it 

Detailed exception control was emplyed for data loading as well.
 explain how and why I did it 



1. new features implementation (Tag & Muscle & Workout Analysis)

% Section - Muscle visualisation

Utilising muscle group vector graphics, enabled to track on muscle status
(wether it is fully recovered, being recovered, or haven't been used for long time)
By using different colour for each muscle state, user can see all muscle status at a glance.

To managing muscle status efficiently, I implemented muscle class.
This class provides the bridge between muscle vector graphics and muscle status data.
I added drawable muscle pathes with each status colour. 
For each muscle group, it checks the saved muscle status and update the displayed muscle graphic's colour.
comparing saved data's muscle name and status, and drawable asset's file name, the app displays the appropriate muscle status.
Additionally, if there is no saved data for muscle, it initialise the muscle data with default values. 
the initial muscle saved data have null value for lastActivity attribute, and the initial status is fully recovered(as there is no workout record added.)

status of muscle is calculated comparing lastActivity attribute in muscle calss and current time.
when user run the app, the app refreshes muscle states checking the current time.
the algorithm detects which msucle status the muscle is currently in.
if the user hasn't worked on that muscle
(empty data on muscle status means user run this app for the first time, 
as when the app initialise the muscles, it puts the lastActivity to null), 
when it's null -> find when the user opened the app and if the time and use it as the last activity 
it assigns the current time to tell user that he hasn't worked on that muscle for long time if he doesn't use that muscle for long time.
if this is not the case, muscle status is determined following logic.
if the user hasn't work on that muscle more 10 days -> this muscle needs to be exercised
else if the user worked on this muscle in recent 2 days -> this muscle needs to be recovered.
otherwise the muscle status is considered as fully recovered.
once the status attributes in muscle data class is updated,
the saved muscle status value can be used to find the appropriate drawable resource.

to enable user to enter any number of muscle groups associated to an exercise when they custom an exercise,
spinner which can select multiple items is employed.
When user enters an exercise log, the main and sub msucle groups associated to that exercise is checked 
and those muscle state are updated to be recovered and the lastActivity attribute will be recovered as well.



% The muscle visualization feature significantly enhances how users interact with and manage their workout progress for each muscle group. The application provides a visual representation of muscle status, such as whether a muscle is fully recovered, in the process of recovery, or has not been used for an extended period. Each state is distinctly colored, enabling users to quickly ascertain the status of their muscles at a glance. This holistic approach not only enriches the user's interaction with the app by providing clear visual cues and updates but also enhances the functionality and accuracy of workout tracking.

% To efficiently manage this feature, I introduced a 'Muscle' class, which serves as a bridge between the muscle vector graphics and the associated muscle status data. This class is crucial for dynamically updating the displayed colors of the muscle graphics based on their current status. It works by comparing the saved muscle data—including names and statuses—with the drawable assets' file names, allowing the app to accurately display the appropriate muscle status for each group. If no data is loaded, the system initializes it with default values, setting the 'lastActivity' attribute to null and the initial status to fully recovered, as this process is done when user run the app for the first time. Which means there will be no workouts have been recorded.

% The app determines muscle status by comparing the 'lastActivity' timestamp in the muscle data with the current time. Upon each app launch, it refreshes the muscle states by checking the current time and applying a logical algorithm to ascertain the muscle's condition. 
% if the lastActivity is null (empty data on lastActivity feild means user run this app for the first time), it puts the lastActivity to null), it assigns the current time for further muscle status tracking. 
% If a muscle has not been exercised for more than ten days, the system flags it as needing exercise; if it has been worked out within the last two days, it indicates that the muscle is in recovery. If none of these conditions apply, the muscle is considered fully recovered. Following this update, the status within the muscle data class is adjusted, and the system then links the updated status to the appropriate drawable resource for visual representation.

% To further tailor the user experience, the application includes a feature allowing users to customize exercises with any number of associated muscle groups. This customization is facilitated by a spinner that can select multiple items, ensuring users can specify both main and sub muscle groups for each exercise log. When an exercise log is entered, the application checks the associated muscle groups and updates their status to 'being recovered', and the lastActivity attribute is updated. 



% section - tag

I implemented tag to provide user ability to customise categories and use this for filtering.
User can add customised tag depends on their needs.
For example, they can use this for customise routines, group specific exercises depends on muscle groups. 

I had two attemptions to succesfully implement this feature.
For the first attempt, I simply used string to deal with each tag.
As there was no specified data structure for tags, it caused problems when I tried to utilise the tags.
For example, users couldn't make two different tags which have identical names.
Additionally, when I wanted to change a tag's colour displaying respect to tags' status (wheather if it is selected or not)
So I tried second implementation attempt, which was used for this version's implementation.
I implemented tag data class for clearer data strucrue and management.
each tag consists of unique id, time added, name, and "isSelected" boolean value which indicates if the tag is selected or not.   
in this class, ADD_TAG and SELECT_TAG are added as "companion object". 

tag bars are added to home fragment and status fragment for the filtering purpose.
ADD_TAG is used for tag customisation, by enable user to add a new tag from clicking on the ADD_TAG on the tag bar.
The tag bar displays all the tags the user has been added. 
It is implemented using horizontal recyclerView. The ADD_TAG is always the last element on this tag bar.
allowing user to add a new tag when they cannot find a tag they want to use.
the filter works by finding selected tags from local storage.
after getting selected tags, the app looks though the ExerciseCards stored, checking wheather or not they contains any of the selected tags.
With this process, only exercise cards which contains at least one of the selected tag is displayed.
when no tag is selected, filter is turned off and all the exercises are displayed. 
the selected tag is kept while main activity's lifecycle.
so, it only is reset when the user opens other activities or turn off the app.
By keeping the tags' isSelected attribute's lifecycle in this way,
users can keep their tag selection across fragments.
Which means, although the user moves from the home fragment to status fragment, or vice versa,
the fragments will display filtered cards reflecting tags user selected.
I implemented this to do not make user have to select tags again and again whenever they move across the fragments.

These tags can be assigned to ExerciseCard when users add an exercise. 
SELECT_TAG is used for desplaying spinner hint by using this as the first tag displayed on the spinner.
by using a spinner which user can select multiple items, user can enter as many tags as they want.


// how does it trigers filtering

[ live data ]
[ TagAdapter ]

\model
tag storage
save tags when it's added
delete and edit tag work


% To enhance user customisation and improve the filtering capabilities of my Android application, I implemented a tagging feature that allows users to create and manage custom categories. Users can add tags based on their preferences, enabling them to customise routines and group specific exercises according to muscle groups or other criteria.

% The initial attempt to implement this feature involved using strings to represent each tag. However, this approach led to several issues, such as the inability to create tags with identical names and difficulties in managing tag properties like colour changes based on their selection status. To address these issues, I developed a more robust solution in my second implementation attempt.

% In this version, I designed a dedicated `Tag` data class to provide a clear structure for tag management. Each tag consists of a unique ID, the time it was added, a name, and a Boolean value (`isSelected`) indicating whether the tag is selected. This structure enables better handling of tags, ensuring that each tag is distinct and manageable. Additionally, `ADD_TAG` and `SELECT_TAG` were added as companion objects within this class to facilitate tag customisation and selection.

% Tag bars were integrated into the home fragment and status fragment to support filtering. The `ADD_TAG` functionality allows users to add new tags by clicking on the `ADD_TAG` item on the tag bar, which is implemented using a horizontal `RecyclerView`. It is always the last element on the tag bar, providing a consistent location for users to add new tags. The tag bar displays all user-added tags, enabling efficient tag management and selection.

% The filtering mechanism works by retrieving selected tags from local storage and then filtering exercise cards based on these tags. If a card contains any of the selected tags, it is displayed; otherwise, it is hidden. When no tags are selected, the filter is disabled, and all exercises are shown. The selected tags persist throughout the main activity's lifecycle, ensuring that users do not need to reselect tags when navigating between fragments. This persistence enhances the user experience by maintaining filter settings across the home and status fragments, allowing seamless transitions without losing tag selections.

% Tags can be assigned to exercise cards when users add exercises. The `SELECT_TAG` companion object is used to display a spinner hint, making it the first tag shown in the spinner. Users can select multiple tags using a multi-select spinner, enabling them to assign numerous tags to an exercise as needed. This comprehensive tagging system significantly enhances the app's customisation and filtering capabilities, demonstrating my proficiency in solving complex problems and implementing effective solutions in Kotlin using Android Studio.



% section - workout analysis 
I provided users workout analysis in a form of graph, with the original data as table.
I employed 1RM to indicate the workout progress on each exercise.
To utilise calculated predicted 1RM, I updated data structure I had.
firstly I added oneRepMax attribute to ExerciseSet.
When user enters an exercise set record, the app automatically calculates the 1RM and assign that value to this attribute.
this is fundamental for handling 1rm value for logs and exercises.
Secondly, I added oneRepMax attribute to ExerciseLog.
Once user finish logging on an exercise, it compares all the oneRepMax value in ExerciseSets entered, and pick the maximum value from it.
this was critical to show workout progress as user can see how their strength is being changed by each session.
Finally, I added oneRepMax attribute to ExerciseCard.
When the app is storing a new exercise log, it checks the stored exercise card's oneRepMax and if the new log contains bigger 1RM, it updates the 1RM for the card.
This enable user to check their personal best record which makes them feel good and motivates them.

the main way user can access to workout analysis is provided by status fragment.
With this status screen, user can see all of exercises' logs they entered in one screen.
Below each exercise card which has exercise name, main and sub muscle groups, the workout analysis is displayed.
this includes the personal best 1RM of the corresponding exercise and the graph shows progress changed.
alongside withe thes personal best, it shows when the user reached that record which ecourage users to keep making further progress. 
below this analysis, user can also check the all the actual original logs they enterd in form of table.


% screenshot 
- > today's log - match parent
the use of an empty row at the end of each session improves visual separation and helps distinguish between sessions.
when add log or card, the most current one to the first index
I will do the add log activity first
add log layout
one rep max with two decimal with kg
one rep max bar
make part bold
new line instead of coloum
implementing the graph
today box : user don't know where to log
table setup object
add log activity keep updating thr logs
grapg visibility
1rm bar visibility
1rm visibility and formatting, muscle text formattin
floating button loaction

analysis is available on the logging screen as well.
this is the noticable unique feature of this app. 
Thanks to this feature, user don't need to go check analysis and logs seperately.
it automatically provides the relavant logs and analysis when user enters their workout records

simple Analysis (1Rm) is also available on home screen. 
For 1RM showed on the each exercise card on home fragment, 
alongside withe thes personal best, it shows when the user reached that record which ecourage users to keep making further progress just like status fragment.
even user don't check the analysis manually with the status screen,
with the home screen and AddLogActivity, analysis are provided to the user which ecourage users to keep making further progress even without checking them manually.

By making status screen as a place where can check the analysis,
there need to be another place for adding customised exercise card.
becaue originally the status screen was responsible for this.
I moved this reponsibility to home fragment screen.
I added the floating button on the home fragment to enable user to enter customised exercise.
For first implementation of thie floating button, it was fixed on the screen, 
but I found this needs to be draggable.
As this was covering the fixed area of the screen which covers the data on the exercise cards.
In this reason, to maximise the limited screen area usage. I made it draggable.
User can drag the button on the screen.
also, to not cover the contents as much as possible,
the button turns into half transperant if the user doesn't touch it for a while.
also the button is aligned to the left or right wall to minimise distraction by covering contents.



log 
status

\predicted 1RM
update existing data structure (added 1RM)
deal with empty list for algorithm
One rep max -> 1rm

data structure update


% Status Fragment 
will change the status layout
implementing the graph
status card
status recycler
table setup object
grapg visibility



(5) feedback / error report / privacy policy

% section - setting 
I implemented the setting screen buttons I added when I was implementing the version 1.5.
Namely there are three buttons I implemented.
1. report error 
2. send feedback 
3. privacy policy 

from this implementation, the setting fragment took the responsibility on communication between users and the app.
By reproting errors and giving feedbacks, user can contribue on app development.
which gives chance for user to use apps which satisfies their requirement better.
this is for accesibility of the app.

reporting error and send feedback button enable user to send an email to the developer.
I made an account which specifically for this app. (trainingtrackermainbox@gamil.com)
when user click the report error or send feedback, they can enter contents they want to send, and then by clicking sending,
user can send this via email.
I needed permission to implement this for an android app
{description on what permission I needed, how I did it, why it is needed.}
Also, I added privacy policy to collect information from user when they are reporting error.
this data collection tries to load data user stored locally,
and report me what data is succesfully loaded.
so I can get better vision on the error. (cuz I can know if this error is generated from failing loading data)

// more reasons why the reporting error and feedback is important and references

% Setting Fragment 
send feedback via email
feedback and reporting handelr


(3) storage sequence adoptation 
    ExerciseCard, ExerciseLog




\\\version 1.5 : 
research on muscle group and visualisation
didn't have 1rm for log, card
delete [
    and analysis available on the logging screen. 
this is the noticable unique feature of this app. 
other exercise apps are separating the analysis screen and the logging screen, 
also, not only giving the past logs, can see the progress at a glance. 
so it is also better than just using excel which simply gives you past logs.
Making analysis is possible with excel as well, but for this, users need to manually setup all the things for each exercise they add.
but with this app, 
]


commit 247e0bf258184c9faf0ffe614508a7354dcc27bf
Date:   Sun Mar 24 02:33:01 2024 +0000
    no need to filter card

commit 9e60ba9b2783710f61608191951fc5bbaa30575c
Date:   Fri Apr 5 18:05:18 2024 +0100
    loop

commit 73e4fbbe2f14968cad2ee2921a684ceec0358e54
Date:   Fri Apr 5 20:18:52 2024 +0100
    can see the cards again

commit 6911b60d767a06c4fe0e25a7049b536db530051f
Date:   Mon Apr 8 16:02:56 2024 +0100
    add app download date

commit 777245ddb2b130e679de96ee2c4d05dcca4bbe56
Date:   Mon Apr 8 15:48:37 2024 +0100
    data map

commit 7c29dacc8f26ecd64c9c3ca0877bfa3f49ed86f5
Date:   Fri Apr 12 20:35:33 2024 +0100
    update out of lambda function

commit 50c353137bc570f1afc35a90af3f737adb651c62
Date:   Fri Apr 12 12:30:43 2024 +0100
    going to update observe lambda


commit 1d1e848cc3b1c889e8a14b5a01d180e539c10ef1
Date:   Fri Apr 12 22:34:46 2024 +0100
    only first set is saved


section - market introduction stage - app deployment (app store release, gitHub release)

for market introduction, I needed a few steps to make my app to be prepared for release.
1. specify the version name and version code for this release. 
2. package domain changed (com.example.trainingtracker -> com.jeein.trainingtracker)
{explanation on what it does, why I need to change it, how my passport name can be suitable for this.}
3. preparing build variant for releasing
by seperating debud(debuggable) / release(not debuggable)
how do I do it and why do I need it?
4. generate signed apk
(build a signed APK that I can deploy to a device, why do I need it/)
-1. steps for setting key store configuration. (how I can get it, why do I need it)
-2. set build variants 


I did app store releasing
I needed to set... 
app title
app SKU
app category
subcategory
language support
customer support contact 
availability (countries / regions)
target audience (age groups)
content rating 
user data privacy
appstore detail
app pricing
description
feature bullets 
keywords
icon
screenshots
promotional image (feature graphic)

for each components, state what it is, how it will be used/why I need it, why it is important. 

github releasing 
I used the promotional image (feature graphic), and privacy policy on readme when I release it on github.
also on release description, I gave the feature bullets with changes I made from the first version. 


conclusion of version 2 implementation

Status fragment is designed to combine the benefit of excel and fitness tracking app.
giving all tje exercies' analysis on one single screen, user can take the benefit of excel which enable people to see all the datas on one screen,
which makes people easily compare their data and see them at a glance.
But also, this gives you the benefit of fitness tracking app.
This includes that
1. user can simply enter their records and don't need to do anything else to see the data and analysis.
2. user can easily filter the exercies they want to see the analysis or logs with the groups they customised.
this is a unique feature provided only on this app which enable user to be able to check the analysis and data on one screen.
By this feature, user don't need to go through and enter each exercise analysis and log to check or compare them.
This solves the reason why people don't use fitness tracking app and use excel instead providing the benefit of excel but also provide even more advantages.
asked people who used the excel instead of fitness app
they said it is indeed helpful and useful feature
and it is even better than excel
as it makes them easily manage the data and it also provide analysis and filter the data.

analysis is available on the logging screen as well.
this is the noticable unique feature of this app. 
other exercise apps are separating the analysis screen and the logging screen, 
also, not only giving the past logs, can see the progress at a glance. 
so it is also better than just using excel which simply gives you past logs.
Making analysis is possible with excel as well, but for this, users need to manually setup all the things for each exercise they add.

feedback agaile reflection


version 3 dissertation

survey result
user feedbacks -> solution


git log categorising
functionality - why did I implement it / how did I implement it?
app scalability / technical explanation



critical evaluation
user feedback
what can I do better
following studies

4. Evaluate the app in terms of efficiency and user satisfaction and demonstrate that the proposed enhancements are beneficial.
	smart goal evaluation
	app reviews
	what I can do better if I do it again next time?