the main aim of this project is increasing accesiblity of the app

found most of people log exercises are using apps or excels
the reason why people are using apps are...
the reason why people are suing excels are...

what if I make the app which has both of their pros and reducing cons?

Also, as the beginner I found it is difficult to check if I am using the correct muscle and track on progress


(2) version 2 implementation
* made the developer account
* implementing the second version of the app
    - focus on user interface (logo design, ui improvement)
    -user friendly ui (picked colour scheme(yellow & blue for colour blind people, intuitive ui for exercise cards and layout, arrangement) )
    - click card to enter log
    - improved card layout
    - show the past records and the personal best record
    - show analysis of the progress (graph – still need to connect it to user’s data, but frontend implementation is done)
    - muscle visualisation (still need to work on colour change features, but I’ve done visualisation each muscle with vector graphics, used “inkscape”)
    - designed app logo
    - Find some errors and fixed (about data structures and classes)
    - enhancing user experiences
(minimalise steps that user needs to take to log their progress, user can customise set recording, long click to delete a card with warning)


2nd version - Improve usability
% git log categorising
% functionality - why did I implement it / how did I implement it?
% app scalability / technical explanation


\version v2.0.0 {
    focus on enhancing existing functionality
    (prepare for additional functionalities)
}


% general UI improvement
added app logo
theme {
    colour scheme added
    -> button theme 
    -> bottom navigation theme 
}


% Home framgment
rather than clicking log button -> make recyclerView clickable
    click home recycler work (when recyclerView item is clicked, it opens up the add log activity)

rather than only enable card management at status fragment, using long click to edit or delete cards - home, long clickable
 
better recyclerView element (card layout, components visibility, card text change)

added tag bar - preparing for a new functionality (will enable filtering with tags entered)


scroll bar - can check how many cards the user have, and where in the list the user at 


    % muscle visualisation {
            - preparing for a new functionality (will enable visualised muscles)
    muscle vectors -> made drawable vector graphics, checked muscle groups and find information about muscles, group them for exercise and visualised them. sketched first and then used a program to make a drawable vector
    going to add muscle map - preparing for a new functionality (to change the muscle colour refelcting the muscle status checking the exercise log user enters, it needed to check which muscle vectors are related to which muscle user added for the exercise card)
    }

% Status Fragment 
\todo : need to check how far the status fragment is implemented with this stage
clickable status recycler 
change status and setting layout
status, long-clickable
status card
Card layout
added tag bar
scroll bar
card text change
Analysis
exercise card item layout changed

one rep max algorithm
one rep max, I wanna use different file for different exercise

[
    the biggest benefit of using excel as logging is that they can check all exercises' log at a glance.
    but for the exercise apps, they need to enter an exercise to check it's progress which is not convenient.
    user need to check all the exercises one by one.
    
    but the biggest disadvantage of using excel is that it is hard to make the detailed analysis organised when they want to check all the data at the glance
    Also, it is difficult to find the exercise when they are doing a lot of exercises. 
    It will be much better if they have filter
    Also, user might want to check analysis for specific category
    this is also difficult for excel
    they can implement it but then it will be difficult to add or delete/edit exercises they are doing
]

% Setting Fragment 
change status and setting layout \todo : need to check how far the setting fragment is implemented


% Add log Avtivity *** this is the main change in this implemented version *****

show the exercise name when logging -> when logging the exercise, user could continuesly check which exercise log they are entering, so they don't forget. also, it makes sure that user is entering logs to the right place. when I was using the app, I found that if I am not sure where I am entering the exercise
changing layout -> when I log the exercise logs, I found I want to check how my progress is being changed right away. it was annoying to separate the place I can check the analysis and enter the log. As then I cannot check it while I am doing exercise, or I need to move around within the app to check and log. So I made the past logs and analysis available on the logging screen. this is also the noticable unique feature of this app. other exercise apps are separating the analysis screen and the logging screen, here also, not only giving the past logs, can see the progress at a glance. so it is also better than just using excel
kg, reps, checkbox, log works -> for warming up sets, also, someone might lift different weights for one set. 
top box - info card layout (for future implementation - this was for easy and intuitive access of analysis)
working on graph layout (for simple, intuitive but effective past log checking and current log checking, I employed excel-like structure (namely table), also this enable user to check all the logs at a glance)
add log layout (planned how I am going to arrange or fit all the views in one screen. as the space is limited, I needed to arrange them efficiently. user shouldn't be distrubted when they are logging, or checking past logs)
log set class (for each log, they have sets. also, each set has weight lifted and reps. each set should have predicted 1rm as well. reflecting real world working out structure, I implemented the set class -> I think I haven't implemented the 1rm attribute for the set yet. need to add about this later)
    exercise card/log package
    exercise log>set
        -> I think these updates are followed from the set class update 
add logset \todo : need to check how it worked, (set doesn't needed to be stored on local storage, so useing the memory to store all the sets user entered, and when the user finish entering the exercise, it automatically upload it to the local storage)

adapter implementation ...ing...
making it look better
table adapter
set style
table row layout for one element / two elements
set, kg, *
table outline
center
serialisable error fi
    -> these are techincal part of the app
    and also about the app ui 

past log
only relavant logs are shown

table day row - need to be debugged
date row debugged, only add log when actually entered values
    table row layout for one element and two elements are used here.
    as the row with date only needs one element 
    and the set row needs two elements
    I implemented them seperately

table without date
forget to remove remove row
table frontend
making seperate rows
formatting date coloum

[
    last activity is now null-able
    when upload the log, it updates the last activity
    in the log, it only store the card id rather than whole card
] - this was for better track on the last activity user entered. As entering the exercise doesn't mean they did the workout, I made the attribute nullable
wrap content for the height and then put recycler view for the remaining page : I wanted to maximise the space for pastlog efficiently when user didn't entered a lot of logs today. So I used the space below the today's log, and it dynamically allocated the space


% Add card Activity
fix add a new card error \todo : need to check what error I fixed
only one line is permitted for the exercise name


% Scalability
serialisable error fix \todo : need to check where I fixed it
layout rename




(3) version 2.1 implementation - app deployment (app store release, gitHub release)
focus on researched factors (user centered app)


\version v2.1.0 {
    adding new functionalities
} 

% Home framgment

% Status Fragment 

% Setting Fragment 


% Add log Avtivity

% Add card Activity

% Scalability







feedback agaile reflection


version 3 dissertation

survey result
user feedbacks -> solution


git log categorising
functionality - why did I implement it / how did I implement it?
app scalability / technical explanation



critical evaluation
user feedback
what can I do better
following studies

4. Evaluate the app in terms of efficiency and user satisfaction and demonstrate that the proposed enhancements are beneficial.
	smart goal evaluation
	app reviews
	what I can do better if I do it again next time?